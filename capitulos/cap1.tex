\chapter{Introdução}
\label{cap:introducao}

As doenças cardiovasculares, que incluem condições como a doença arterial coronariana, hipertensão arterial, insuficiência cardíaca e acidente vascular cerebral, exercem um impacto significativo não apenas na qualidade de vida dos indivíduos em termos físicos, mas também geram repercussões de grande relevância nos âmbitos social, econômico e de saúde \cite{Stevens2018}.

Segundo dados do relatório \textit{Global Burden of Disease} 2019, as doenças cardiovasculares assumem a posição de destaque como a principal causa de morte em todo o mundo. Esse fato ilustra a magnitude do desafio que essas condições representam para a saúde global \cite{https://doi.org/10.6069/1d4y-yq37}.

No cenário brasileiro, as implicações econômicas se destacam, uma vez que os gastos relacionados às doenças cardiovasculares  é responsável por 28\% de todos os óbitos ocorridos no Brasil. Os custos estimados das doenças cardiovasculares foram de R\$ 37,1 bilhões em 2015. Os custos estimados de morte prematura por doenças cardiovasculares representam 61\% do custo total das doenças cardiovasculares, os custos diretos com internações e consultas foram de 22\% e os custos relacionados à perda de produtividade relacionada à doença foram de 15\% do total. Os gastos em saúde no Brasil são estimados em 9,5\% do PIB e o custo médio das doenças cardiovasculares foi estimado em 0,7\% do PIB
\cite{Siqueira2017}. Esse resultados ressalta a importância de medidas eficazes de prevenção e controle dessas enfermidades, não apenas para a saúde da população, mas também para a estabilidade econômica do país.

Dentro desse contexto, a introdução da ecocardiografia na década de 70 representou um marco significativo no diagnóstico das doenças cardiovasculares. A ecocardiografia trouxe avanços substanciais, permitindo uma abordagem mais precisa e eficaz \cite{haley_2018}. Esta técnica, conhecida por ser de baixo custo e minimamente invasiva, tornou-se amplamente acessível aos pacientes, oferecendo informações cruciais para o manejo de uma ampla variedade de condições cardiovasculares \cite{Mancuso2014}.

 \textcite{Mancuso2014} descreve a ecocardiografia como um papel fundamental como ferramenta multidisciplinar no diagnóstico e na avaliação prognóstica de síndromes coronarianas agudas, insuficiência cardíaca e outras afecções cardíacas. Além disso, ela é essencial para determinar o tratamento adequado em pacientes com arritmias cardíacas. Sua importância na medicina cardiovascular é indiscutível, contribuindo para a melhoria da qualidade de vida dos pacientes e o aprimoramento da gestão clínica dessas condições.

Os dispositivos portáteis de ecocardiografia são capazes de examinar a funcionalidade cardíaca dos pacientes fora das clínicas,a capacidade de fornecer imagens em tempo real é uma vantagem importante desses scanners de ecocardiografia \cite{Xiao}. Apesar de sua eficiência, a interpretação dos resultados ainda depende da experiência  do operador, o que pode resultar em variações e possíveis erros diagnósticos. A fim de contornar esse problema, a Inteligência Artificial (IA) tem sido proposta como uma ferramenta para tornar as interpretações dos ecocardiogramas mais precisas, coerentes e automatizadas.A aplicação da IA na ecocardiografia pode minimizar o risco de erros humanos e contribuir significativamente para a melhoria da qualidade do diagnóstico médico \cite{Alsharqi2018}.

Uma das métricas fundamentais para o diagnóstico de patologias cardíacas é a Fração de Ejeção do Ventrículo Esquerdo (FEVE), que possui um valor prognóstico significativo na previsão de desfechos adversos em pacientes com insuficiência cardíaca \cite{kosaraju_grigorova_goyal_makaryus}. A extração precisa da FEVE é desafiadora devido à necessidade de seleção de quadros de alta qualidade durante as fases de Diástole (D) e Sístole (S), um processo suscetível a erros dada a variabilidade inter e intra-observador \cite{Ouyang2020}.

O avanço no campo de aprendizado de máquina aplicado à análise de imagens biomédicas tem sido notável. Contudo, o tratamento de fluxos de vídeo de ecocardiograma enfrenta desafios específicos, como a escassez de dados de vídeo médico que sejam abertamente disponíveis e adequadamente anotados, o que representa um obstáculo para o desenvolvimento de técnicas de análise mais precisas \cite{Ouyang2020}.

Diante desses desafios, este estudo tem como objetivo principal melhorar a precisão diagnóstica em cardiologia, proporcionar suporte ao operador e atender de forma eficaz às necessidades clínicas de pacientes variados, assegurando melhorias na qualidade do diagnóstico cardíaco. Para alcançar este objetivo, propõe-se uma revisão sistemática para identificar métodos eficazes e lacunas na medição da Fração de Ejeção do Ventrículo Esquerdo (FEVE). A partir dos resultados obtidos, selecionamos um modelo com potencial para refinamento. Este aprimoramento se baseia em identificar e superar as limitações específicas do modelo selecionado, aumentando assim a precisão e a generalização na estimativa da FEVE. Adicionalmente, a validade e a eficácia do modelo aprimorado são testadas através da análise de dados clínicos de pacientes adultos e pediátricos, para garantir sua aplicabilidade em diversos contextos clínicos.

A principal contribuição desta dissertação é a introdução e aplicação de uma metodologia  de avaliação e experimentação utilizando modelos de neurais artificias na análise de ecocardiogramas. Propõe-se adaptações específicas na arquitetura das redes neurais existentes, juntamente com a otimização criteriosa dos hiperparâmetros, para melhorar a precisão e a robustez na estimativa da Fração de Ejeção do Ventrículo Esquerdo (FEVE). A abordagem detalha o desenvolvimento de uma pipeline automatizada que incorpora pré-processamento de dados, treinamento de modelos, validação cruzada e análise de desempenho, garantindo que os modelos treinados não só sejam precisos, mas também generalizáveis a diversos contextos clínicos.

Esta dissertação está organizada da seguinte maneira: o Capítulo \ref{Fundamentação teórica} aborda temas cruciais em cardiologia e tecnologia, incluindo anatomia e fisiologia do coração, ultrassonografia, ecocardiograma e introdução às redes neurais artificiais. O Capítulo \ref{Revisão bibliográfica} explora a aplicação de redes neurais profundas na avaliação da função cardíaca utilizando vídeos de ecocardiograma, descrevendo a metodologia de revisão sistemática da literatura e discutindo estudos selecionados. O Capítulo \ref{Metodologia} detalha o processo metodológico para aprimorar a análise de ecocardiogramas, abordando a descrição e justificativas dos experimentos, a descrição das bases de dados, a implementação da arquitetura modificada e a otimização de parâmetros. O Capítulo \ref{sec
} apresenta e discute os resultados dos experimentos realizados com os modelos clássicos e modificados, testando a hipótese de generalização dos modelos. Por fim, o Capítulo \ref{sec
ão} de conclusão revisa os principais achados da pesquisa, destacando a melhoria na precisão da análise de ecocardiogramas para o cálculo da fração de ejeção do ventrículo esquerdo (FEVE), discutindo as implicações clínicas, as limitações do estudo e sugerindo direções para futuros trabalhos.



