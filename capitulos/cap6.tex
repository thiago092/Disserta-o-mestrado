\chapter{Conclusão}
\label{sec:conclusão}

Este trabalho investigou a análise de ecodardiogramas e o calculo automático da FEVE usando de redes generativas transformers. O objetivo principal foi desenvolver e validar uma arquitetura modificada de aprendizado de máquina, baseada na integração de técnicas avançadas de processamento de imagens e modelos de aprendizado profundo, visando melhorar a precisão e eficiência dos diagnósticos cardíacos.

A metodologia adotada incluiu uma revisão sistemática da literatura, seguida pela seleção e preparação de conjuntos de dados de ecocardiogramas adultos e pediátricos. Foram implementadas modificações na arquitetura UViT clássica, integrando elementos de redes neurais profundas como RoBERTa para analisar as características espaço-temporais dos vídeos de ultrassom cardíaco. A eficácia desta abordagem foi testada através de uma série de experimentos que compararam o modelo modificado com a versão clássica em termos de precisão na estimativa da fração de ejeção.

Os resultados demonstraram uma melhoria no desempenho com a arquitetura modificada, especialmente com o uso de RoBERTa, que alcançou uma maior precisão e um melhor coeficiente de determinação (R²) em comparação com o modelo original. A análise dos dados combinados de pacientes adultos e pediátricos evidenciou que, apesar de um aumento nos valores de erro médio absoluto (MAE) e erro quadrático médio (RMSE), o coeficiente R² melhorou, indicando uma maior capacidade do modelo em explicar a variabilidade dos dados.

As principais contribuições deste estudo incluem o desenvolvimento de uma abordagem robusta para análise de ecocardiogramas, que combina técnicas de redução de dimensionalidade como a ResNet Autoencoder (ResNetAE) e processamento espaço-temporal para oferecer diagnósticos mais precisos. Além disso, a capacidade de generalização do modelo para adultos, crianças e condições patológicas representa um avanço importante para a prática clínica, facilitando uma avaliação mais precisa e confiável da função cardíaca.

Para os trabalhos futuros, a pesquisa pode se beneficiar da exploração de modelos que categorizem os pacientes por idade para ajustar automaticamente a análise da fração de ejeção com base nas características específicas de cada faixa etária. Além disso, a integração de modelos de linguagem avançados, como GPT-4, pode ser explorada para aprimorar a análise de textos clínicos e relatórios de imagem, oferecendo um entendimento mais profundo das narrativas médicas e contribuindo para uma melhor interpretação dos resultados dos ecocardiogramas. Outras técnicas promissoras incluem o uso de redes generativas adversárias (GANs) para aprimorar a qualidade das imagens de ultrassom e a implementação de técnicas de aprendizado semi-supervisionado e não supervisionado para lidar com a escassez de dados anotados de alta qualidade. Essas abordagens podem proporcionar melhorias significativas na automatização e precisão dos diagnósticos de condições cardíacas, representando um potencial considerável para futuras investigações na área de imagens médicas. Adicionalmente, uma etapa crucial será a validação clínica dos modelos desenvolvidos, envolvendo testes em contextos clínicos reais para avaliar sua eficácia e segurança, visando sua implementação prática e rotineira na análise ecocardiográfica.